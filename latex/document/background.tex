\chapter{Contesto}

Come anticipato nell'introduzione, il contesto in cui si va a collocare il
lavoro di tesi è molteplice. Abbiamo infatti a che fare con algoritmi genetici,
calcolo parallelo ed explainable AI.

\section{Algoritmi genetici}

Gli algoritmi genetici fanno parte delle possibili tecniche euristiche adottate
nell'ambito dell'ottimizzazione. Forniscono infatti soluzioni subottimali ma
in tempi generalmente inferiori rispetto ad algoritmi in grado di trovare
soluzioni ottime, ma con complessità più elevata.

In contesti in cui non è necessario trovare la soluzione ottima ma si cerca
solo di minimizzare (o massimizzare) una certa funzione obbiettivo, gli
algoritmi genetici sono una valida scelta.

Il loro comportamento simula l'evoluzione di una popolazione di individui, i
quali rappresentano possibili soluzioni del problema, tramite paradigmi
evolutivi in cui gli individui migliori sopravvivono e si riproducono,
generando nuovi individui che condividono parte del genoma dei genitori.

Ogni individuo è rappresentato da un cromosoma, il quale ha generalmente la
forma di una lista o di un vettore, i cui attributi costituiscono una possibile
soluzione al problema o una sua codifica. Nella loro forma più comune sono
composti da sei fasi:

\begin{enumerate}
	\item Generazione: viene generata in modo casuale una popolazione di
	      cromosomi i cui attributi appartengono al dominio del problema.
	\item Selezione: Vengono selezionati i $k$ migliori individui di una certa
	      generazione.
	\item Crossover: due o più individui si scambiano il materiale genetico,
	      ossia gli attributi dei cromosomi, generando nuovi individui.
	\item Mutazione: viene applicata una piccola variazione al cromosoma in
	      favorendo l'esplorazione di più valori nello spazio delle soluzioni.
	\item Valutazione: ogni individuo viene valutato in base ad una funzione di
	      che gli attribuisce un valore di \textit{fitness}.
	\item Rimpiazzo: si decide quali individui tenere e quali scartare tra una
	      generazione e l'altra.
\end{enumerate}

Tutte le fasi tranne la prima sono eseguite in un ciclo che termina quando
viene soddisfatto un criterio di convergenza. Le iterazioni del ciclo
rappresentano le generazioni della popolazione, le quali differiscono per
individui che compongono la popolazione, diversità genetica e valore medio di
fitness. Abbiamo quindi una struttura di algoritmo genetico di questo tipo:

\begin{center}
	\includesvg[inkscapelatex=false, scale=0.8]{immagini/genetic_algorithm.svg}
\end{center}

che è la stessa implementata per il lavoro di tesi e il cui criterio di
convergenza è un limite al numero di iterazioni massime.

\subsection{Calcolo parallelo}

Uno degli obbiettivi principali è stato quello di riuscire a sfruttare
architetture multi-core per svolgere alcune delle fasi dell'algoritmo genetico
in parallelo. Nel caso di una struttura di algoritmo genetico come quella
precedentemente descritta, si è proposto un modello di calcolo parallelo di
questo genere

\begin{center}
	\includesvg[inkscapelatex=false, scale=0.65]{immagini/map_reduce.svg}
\end{center}

che mira ad implementare un paradigma \textit{map} per le fasi di crossover,
mutazione e valutazione. Generalmente la valutazione costituisce la fase più
dispendiosa in termini di tempo, ma anche crossover e mutazione posso essere
motivo rallentamenti. Aggiungiamo inoltre che tutte e tre le fasi si prestano
bene allo stesso tipo di operazione, con la differenza che non a tutti gli
individui verranno applicati gli operatori di crossover e mutazione.

\section{Explainable AI}

L'explainable AI è una branca della ricerca che cerca di fornire metodi per
l'interpretazione di modelli di machine learning definiti \textit{blackbox},
così definiti poiché talmente complessi da risultare impossibili da comprendere
per un essere umano. Tra i modelli definiti blackbox abbiamo le reti neurali
e le foreste casuali, i quali sono stati impiegati per la fase di test del
progetto.

Di questi modelli si conosce l'architettura, il funzionamento e il modello
matematico che li descrive. Ma le implementazioni finiscono spesso per avere
una quantità di parametri talmente elevata da rendere impossibile capire come
e quanto i singoli parametri influiscano nelle decisioni e nelle predizioni
fatte dal modello.

In altre parole, una volta che il modello effettua delle predizioni o prende
delle decisioni non è possibile capire il motivo di tali scelte. Non è infatti
possibile capire quale parte del processo abbia portato a quella risposta, per
esempio un dataset di training con errori o una cattiva scelta degli
iperparametri.

Si è quindi iniziato a ricercare metodi per rendere le scelte effettuate dai
modelli \textit{spiegabili} e comprensibili da un essere umano. Tra questi
è presente \textit{LORE}~\cite{guidotti2018LORE}, che si propone di generare
spiegazioni locali a singole decisioni prese dai modelli. \textit{LORE} non
cerca infatti di comprendere il funzionamento globale del modello, ma va ad
analizzare le singole decisioni prese su una singola istanza dei dati in input.

Mentre altri metodi puntano ad una comprensione globale del modello, per
esempio cercando di definire i confini di classificazione, \textit{LORE} lavora
sulle singole istanze andando a ridurre la complessità di tale confine e
riuscendo ad ottenere spiegazioni più accurate.

Il modo in cui opera \textit{LORE} prevede l'esecuzione di un algoritmo
genetico su ognuna delle istanze classificate dal modello e di cui si vuole
generare una spiegazione. In una prima fase si cerca vuole generare un vicinato
sintetico i cui individui siano classificati allo stesso modo dell'istanza
di riferimento. Si cerca poi di generare individui sintetici classificati in
modo diverso da essa, cercando in entrambi i casi di minimizzare la distanza
tra i vicini sintetici e l'istanza.

Una volta generati, questi individui vengono usati per allenare un albero di
decisione. L'albero risultante conterrà delle regole che andranno poi a
costituire la spiegazione della scelta fatta. Un albero decisionale è infatti
rappresentabile come un vero e proprio albero, le cui diramazioni corrispondono
ai valori soglia di determinate feature i quali porteranno ad una scelta
piuttosto che ad un'altra.
