\chapter{Contesto}

Come anticipato nell'introduzione, il contesto in cui si va a collocare il
lavoro di tesi è molteplice. Abbiamo infatti a che fare con algoritmi genetici,
calcolo parallelo ed explainable AI.

\section{Algoritmi genetici}

Gli algoritmi genetici fanno parte delle possibili tecniche euristiche adottate
nell'ambito dell'ottimizzazione. Forniscono infatti soluzioni subottimali ma
in tempi generalmente inferiori rispetto ad algoritmi in grado di trovare
soluzioni ottime, ma con complessità più elevata.

In contesti in cui non è necessario trovare la soluzione ottima ma si cerca
solo di minimizzare (o massimizzare) una certa funzione obbiettivo, gli
algoritmi genetici sono una valida scelta.

Il comportamento di questi algoritmi tende a simulare l'evoluzione di una
popolazione di individui, i quali rappresentano possibili soluzioni, tramite
meccanismi basati sulla fitness. Ogni individuo è rappresentato da un cromosoma,
il quale ha generalmente la forma di una lista o di un vettore, i cui attributi
costituiscono la soluzione al problema o una codifica di essa. Nella loro forma
più comune sono composti da sei fasi:

\begin{enumerate}
	\item Generazione:
	\item Selezione:
	\item Crossover:
	\item Mutazione:
	\item Valutazione:
	\item Rimpiazzo:
\end{enumerate}
