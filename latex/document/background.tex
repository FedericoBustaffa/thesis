\chapter{Contesto}

Come anticipato nell'introduzione, il contesto in cui si va a collocare il
lavoro di tesi è molteplice. Abbiamo infatti a che fare con algoritmi genetici,
calcolo parallelo ed explainable AI.

\section{Algoritmi genetici}

Gli algoritmi genetici fanno parte delle possibili tecniche euristiche adottate
nell'ambito dell'ottimizzazione. Forniscono infatti soluzioni subottimali ma
in tempi generalmente inferiori rispetto ad algoritmi in grado di trovare
soluzioni ottime, ma con complessità più elevata.

In contesti in cui non è necessario trovare la soluzione ottima ma si cerca
solo di minimizzare (o massimizzare) una certa funzione obbiettivo, gli
algoritmi genetici sono una valida scelta.

Il comportamento di questi algoritmi tende a simulare l'evoluzione di una
popolazione di individui, i quali rappresentano possibili soluzioni, tramite
meccanismi basati sulla fitness. Ogni individuo è rappresentato da un cromosoma,
il quale ha generalmente la forma di una lista o di un vettore, i cui attributi
costituiscono la soluzione al problema o una codifica di essa. Nella loro forma
più comune sono composti da sei fasi:

\begin{enumerate}
	\item Generazione: viene generata in modo casuale una popolazione di
	      cromosomi i cui attributi appartengono al dominio del problema.
	\item Selezione: Vengono selezionati i $k$ migliori individui di una certa
	      generazione.
	\item Crossover: due o più individui si scambiano il materiale genetico,
	      ossia gli attributi dei cromosomi, generando nuovi individui.
	\item Mutazione: viene applicata una piccola variazione al cromosoma in
	      favorendo l'esplorazione di più valori nello spazio delle soluzioni.
	\item Valutazione: ogni individuo viene valutato in base ad una funzione di
	      che gli attribuisce un valore di \textit{fitness}.
	\item Rimpiazzo: si decide quali individui tenere e quali scartare tra una
	      generazione e l'altra.
\end{enumerate}

Tutte le fasi tranne la prima sono eseguite in un ciclo che termina quando
viene soddisfatto un criterio di convergenza. Le iterazioni del ciclo
rappresentano le generazioni della popolazione, le quali differiscono per
individui che compongono la popolazione, diversità genetica e valore medio di
fitness. Abbiamo quindi una struttura di algoritmo genetico di questo tipo:

\begin{center}
	\includesvg[inkscapelatex=false, scale=0.8]{immagini/genetic_algorithm.svg}
\end{center}

che è la stessa implementata per il lavoro di tesi e il cui criterio di
convergenza è un limite al numero di iterazioni massime.

\subsection{Calcolo parallelo}

Uno degli obbiettivi principali è stato quello di riuscire a sfruttare
architetture multi-core per svolgere alcune delle fasi dell'algoritmo genetico
in parallelo. Nel caso di una struttura di algoritmo genetico come quella
precedentemente descritta, si è proposto un modello di calcolo parallelo di
questo genere

\begin{center}
	\includesvg[inkscapelatex=false, scale=0.65]{immagini/map_reduce.svg}
\end{center}

che mira ad implementare un paradigma \textit{map} per le fasi di crossover,
mutazione e valutazione.
