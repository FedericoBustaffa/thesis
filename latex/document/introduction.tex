\chapter*{Introduzione}

Il lavoro di tesi verte sull'implementazione di una libreria di algoritmi
genetici, in grado di sfruttare architetture multicore per il calcolo parallelo.
La necessità di un algoritmo genetico con questa struttura nasce in un contesto
di explainable AI, in particolare si fa riferimento al progetto
\textit{LORE}~\cite{guidotti2018LORE}, il quale utilizza un'implementazione
fornita dalla libreria di algoritmi genetici \textit{DEAP}~\cite{fortin2012DEAP}.

Il metodo \textit{LORE} si propone di generare spiegazioni a decisioni o
predizioni fatte da modelli di machine learning, spesso difficili da
interpretare per via della loro elevata complessità, come ad esempio reti
neurali profonde o forest casuali. In una prima fase, viene utilizzato un
algoritmo genetico su ogni istanza del dataset per la generazione di dati
sintetici con determinate caratteristiche, fondamentali in seguito per la
costruzione delle spiegazioni.

Il problema principale di questo approccio sono le performace. Risulta infatti
essere un metodo molto dispendioso soprattutto quando si ha a che fare con
dataset grandi, si vogliono generare molti punti sintetici oppure il modello
per la classificazione è particolarmente lento in fase di predizione.

Scendendo più in dettaglio, le implementazioni di \textit{LORE} e \textit{DEAP},
così come la maggior parte delle librerie di machine learning, sono
implementate (o forniscono una API) in Python, rendendo quindi necessario
implementare (o fornire una API) nello stesso linguaggio.

Durante la prima fase del lavoro si sono quindi esplorate possibili soluzioni
per un'implementazione in grado lavorare in sinergia con le altre componenti in
gioco e che fosse più performante possibile. Essendo però Python il linguaggio
scelto, non è possibile un'implementazione multithread a causa del GIL (Global
Interpreter Lock), il quale non permette l'esecuzione simultanea di thread con
task ti tipo CPU-bound. Stanno tuttavia emergendo possibili soluzioni a tale
problema, diventate poi oggetto di ricerca per la tesi.

Una volta implementata la libreria si è passati ad una fase di test per
valutarne correttezza e qualità, impiegando l'algoritmo per la risoluzione di
problemi ben noti in letteratura e che si potessero rappresentare facilmente
tramite in modo grafico. Tra questi abbiamo il problema dello zaino, del
commesso viaggiatore e un semplice caso di regressione lineare. Infine è stata
testata la correttezza sul caso d'uso precedentemente descritto per \textit{LORE}.

% Una volta eseguiti test qualitativi sul correttezza e qualitò delle soluzioni
% prodotte, si è passati ad un'esplorazione più approfondita nella quale si è 

Dato che anche \textit{DEAP} fornisce la possibilità di sfruttare il
parallelismo tramite l'impiego di processi multipli, nell'ultima fase sono
stati effettuati dei benchmark sulle performance, mettendo a confronto i
tempi impiegati dalle due librerie al variare di diversi parametri come numero
di \textit{worker}, dimensione dell'input e modello utilizzato per la
classificazione.

I risultati ottenuti superano DEAP nella maggior parte dei casi, fornendo tempi
d'esecuzione inferiori e dimostrando una maggiore capacità di sfruttare
architetture multicore quando si tratta di calcolo intensivo.