\chapter*{Introduzione}

Il lavoro di tesi verteva sull'implementazione di una libreria di algoritmi
genetici, in grado di sfruttare architetture multicore per il calcolo parallelo.

Tale esigenza nasce dalla necessità di ridurre i tempi di esecuzione in un
contesto di \textit{explainable AI}. Nello specifico si fa riferimento a
\textit{LORE}, un framework per la generazione di spiegazioni delle scelte
fatte da modelli di machine learning. La prima fase dell'algoritmo sfrutta un
algoritmo genetico per la generazione di punti sintetici con determinate
caratteristiche, utilizzati in seguito per la costruzione di un albero di
decisione.

Sia \textit{LORE} che la maggior parte di librerie di machine learning
forniscono un'implementazione Python. Da qui la necessità di avere una API
Python anche per la libreria implementata. La prima fase del lavoro è stata
quindi incentrata sull'esplorazione di possibili soluzioni per
un'implementazione in grado di sfruttare il parallelismo e che fosse
utilizzabile tramite una API Python.

Si è poi passato ad una fase di benchmark in cui si sono valutate le
performance e si è effettuato un confronto con la libreria \textit{DEAP},
attualmente impiegata da \textit{LORE}~\cite{guidotti2018survey}.