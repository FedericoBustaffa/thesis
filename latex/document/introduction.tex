\chapter*{Introduzione}

Il lavoro di tesi verteva sull'implementazione di una libreria di algoritmi
genetici, in grado di sfruttare architetture multicore per il calcolo parallelo.

Tale esigenza nasce dalla necessità di ridurre i tempi di esecuzione in un
contesto di \textit{explainable AI}. Nello specifico si fa riferimento al
progetto \textit{LORE}~\cite{DBLP:journals/corr/abs-1805-10820} per la
generazione di spiegazioni delle decisioni prese da un modello di machine
learning. La prima fase dell'algoritmo sfrutta infatti un algoritmo genetico
implementato dal framework \textit{DEAP}~\cite{DEAP_JMLR2012} per la
generazione di punti sintetici con determinate caratteristiche, utilizzati in
seguito per la costruzione delle \textit{spiegazioni}.

Sia \textit{LORE} che la maggior parte di librerie di machine learning
forniscono un'implementazione Python. La prima fase del lavoro è stata quindi
incentrata sull'esplorazione di possibili soluzioni per un'implementazione in
grado di sfruttare il parallelismo e che fosse utilizzabile tramite una API
Python.

Si è poi passati ad una fase di test su problemi ben noti, come il problema
dello zaino e del commesso viaggiatore, per verificare qualità e correttezza
dell'implementazione proposta. Si è inoltre valutata la correttezza delle
versioni sequenziale e parallela sul problema di riferimento.

Una volta testata la correttezza dell'implementazione è seguita una fase di
benchmark in cui si sono valutate le performance e il grado di parallelismo
raggiunto. Dato che anche \textit{DEAP} fornisce la possibilità di sfruttare il
parallelismo, si è deciso di confrontare le performance dei due framework sia
nelle loro versioni sequenziali che parallelizzate.

Infine sono stati svolti dei test sulla qualità dei risultati prodotti. Qui
l'obbiettivo è stato quello di verificare che l'implementazione proposta fosse
in grado di produrre risultati comparabili con quelli ottenuti tramite
\textit{DEAP}.