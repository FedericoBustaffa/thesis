\chapter*{Introduzione}

Il lavoro di tesi verteva sull'implementazione di una libreria di algoritmi
genetici, in grado di sfruttare architetture multicore per il calcolo parallelo.

Tale esigenza nasce dalla necessità di ridurre i tempi di esecuzione in un
contesto di \textit{explainable AI}. Nello specifico si fa riferimento al
progetto \textit{LORE}~\cite{guidotti2018lore}, un framework per la generazione
di spiegazioni delle decisioni prese da un modello di machine learning. La prima
fase dell'algoritmo frutta infatti un algoritmo genetico per la generazione di
punti sintetici con determinate caratteristiche, utilizzati in seguito per la
costruzione di un albero di decisione, fondamentale per fornire le
\textit{spiegazioni}.

Sia \textit{LORE} che la maggior parte di librerie di machine learning
forniscono un'implementazione Python. La prima fase del lavoro è stata quindi
incentrata sull'esplorazione di possibili soluzioni per un'implementazione in
grado di sfruttare il parallelismo e che fosse utilizzabile tramite una API
Python.

Si è poi passati ad una fase di benchmark in cui si sono valutate le
performance e si è effettuato un confronto con la libreria
\textit{DEAP}~\cite{DEAP_JMLR2012}, attualmente impiegata da \textit{LORE}.

Infine è stato svolto un benchmark sulla qualità dei risultati prodotti. Qui
l'obbiettivo è stato quello di verificare che l'implementazione proposta fosse
in grado di produrre risultati comparabili con quelli ottenuti con tramite
il modulo \textit{DEAP}.