\chapter*{Introduzione}

Il lavoro di tesi verte sull'implementazione di una libreria di algoritmi
genetici, in grado di sfruttare architetture multicore per il calcolo parallelo.
L'esigenza di implementare un algoritmo con queste caratteristiche nasce dalla
necessità di ottimizzare un precedente lavoro, il quale ne fa largo uso.

Il progetto in questione è \textit{LORE}~\cite{guidotti2018LORE}, il quale si
colloca nell'ambito dell'explainable AI e che ha quindi l'obbiettivo di
\textit{spiegare} le decisioni prese da un modello di machine learning.
Attualmente \textit{LORE} sfrutta \textit{DEAP}~\cite{fortin2012DEAP}, come
libreria di algoritmi genetici, la quale offre un alto livello di modularità e
flessibilità, mettendo a disposizione dell'utente la possibilità di sfruttare
il parallelismo, dando però poco controllo sull'architettura del modello di
calcolo.

Si è quindi cercato di implementare un'architettura più ottimizzata e specifica,
in g

Le implementazioni di \textit{LORE} e \textit{DEAP}, così come la maggior parte
delle librerie di machine learning forniscono una API Python, rendendo quindi
necessario implementare una libreria con una API nello stesso linguaggio.

Per valutare qualità e correttezza sono stati svolti test su problemi classici,
come il problema del commesso viaggiatore e dello zaino o, per valutare il
comportamento su problemi a valori reali, un semplice caso di regressione
lineare. Tutti problemi in cui la qualità della soluzione generata fosse
facilmente verificabile e rappresentabile graficamente.

Una volta implementata la libreria, si è passati ad una fase di test in cui
si sono valutate e confrontate qualità e performace con \textit{DEAP}.

Infine sono stati svolti dei test sulla qualità dei risultati prodotti. Qui
l'obbiettivo è stato quello di verificare che l'implementazione proposta fosse
in grado di produrre risultati comparabili con quelli ottenuti da \textit{DEAP}.