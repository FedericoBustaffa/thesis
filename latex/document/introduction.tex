\chapter*{Introduzione}

Il lavoro di tesi verte sull'implementazione di una libreria di algoritmi
genetici, in grado di sfruttare architetture multicore per il calcolo parallelo.

Tale esigenza nasce dalla necessità di ridurre i tempi di esecuzione in un
contesto di \textit{explainable AI}. Nello specifico si fa riferimento al
progetto \textit{LORE}~\cite{DBLP:journals/corr/abs-1805-10820}, un metodo per
fornire spiegazioni a decisioni o predizioni fatte da una modello di machine
learning. In una prima fase \textit{LORE} sfrutta infatti un algoritmo genetico,
la cui implementazione è attualmente fornita dalla libreria
\textit{DEAP}~\cite{DEAP_JMLR2012}. La libreria in questione garantisce un alto
livello di modularità e varietà, dando all'utente la possibilità di comporre e
personalizzare il proprio algoritmo come meglio crede. Può inoltre sfruttare il
parallelismo tramite l'utilizzo di processi multipli ma non fornisce molto
controllo sull'implementazione del paradigma di calcolo.

Dato che \textit{LORE} e gran parte delle librerie di machine learning sono
scritte in Python, al fine di garantire una sinergia tra i sistemi si è deciso
di optare per l'implementazione di un qualcosa che avesse una API nello stesso
linguaggio. Sono state quindi vagliate varie opzioni allo scopo di soddisfare
tali esigenze, optando alla fine per un'implementazione interamente in Python.

Per valutare qualità e correttezza sono stati svolti test su problemi classici,
come il problema del commesso viaggiatore e dello zaino o, per valutare il
comportamento su problemi a valori reali, un semplice caso di regressione
lineare. Tutti problemi in cui la qualità della soluzione generata fosse
facilmente verificabile e rappresentabile graficamente.

Una volta implementata la libreria, si è passati ad una fase di test in cui
si sono valutate e confrontate qualità e performace con \textit{DEAP}.

Infine sono stati svolti dei test sulla qualità dei risultati prodotti. Qui
l'obbiettivo è stato quello di verificare che l'implementazione proposta fosse
in grado di produrre risultati comparabili con quelli ottenuti da \textit{DEAP}.