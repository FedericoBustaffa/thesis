\chapter*{Introduzione}

Il lavoro di tesi verte sull'implementazione di una libreria di algoritmi
genetici, in grado di sfruttare architetture multicore per il calcolo parallelo.

Tale esigenza nasce dalla necessità di ridurre i tempi di esecuzione in un
contesto di \textit{explainable AI}. Nello specifico si fa riferimento al
progetto \textit{LORE}~\cite{DBLP:journals/corr/abs-1805-10820}, un metodo per
fornire la spiegazioni alle decisioni o predizioni fatte da una modello di
machine learning. In una prima fase \textit{LORE} sfrutta infatti un algoritmo
genetico, la cui implementazione è attualmente fornita dalla libreria
\textit{DEAP}~\cite{DEAP_JMLR2012}. Questa libreria garantisce un alto livello
di modularità e varietà, dando all'utente la possibilità di personalizzare il
proprio algoritmo come meglio crede. Può inoltre sfruttare il parallelismo
tramite l'utilizzo di processi multipli ma non fornisce molto controllo
sull'implementazione del paradigma di calcolo.

Dato che tutti i framework citati fino ad ora, così come la maggior parte delle
librerie di machine learning, sono scritte in Python, si è scelto di produrre
qualcosa necessità di avere l'interfaccia nel medesimo linguaggio.

La prima fase del lavoro è stata infatti quella riguardante l'esplorazione di
possibili soluzioni per implementare un paradigma di calcolo parallelo in
Python.

Si è poi passati ad una fase di test su problemi ben noti, come il problema
dello zaino e del commesso viaggiatore, per verificare qualità e correttezza
dell'implementazione proposta. Si è inoltre valutata la correttezza delle
versioni sequenziale e parallela sul problema di riferimento.

Una volta testata la correttezza dell'implementazione è seguita una fase di
benchmark in cui si sono valutate le performance e il grado di parallelismo
raggiunto. Dato che anche \textit{DEAP} fornisce la possibilità di sfruttare il
parallelismo, si è deciso di confrontare le performance dei due framework sia
nelle loro versioni sequenziali che parallelizzate.

Infine sono stati svolti dei test sulla qualità dei risultati prodotti. Qui
l'obbiettivo è stato quello di verificare che l'implementazione proposta fosse
in grado di produrre risultati comparabili con quelli ottenuti tramite
\textit{DEAP}.