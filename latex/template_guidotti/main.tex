\documentclass[11pt]{report} % Dimensione testo 11
% Margini e interlinea
\usepackage[top=1in, bottom=1in, left=1.2in, right=1in]{geometry}
\usepackage[english]{babel} % Languange
\pagestyle{plain}
\linespread{1.5}
\usepackage[utf8]{inputenc}
\usepackage[nouppercase]{frontespizio} % Frontespizio
\usepackage{afterpage} % Pagine vuote
\usepackage{url} % Link
\usepackage[hidelinks]{hyperref}
\usepackage{amsmath} % Typesetting mathematical notation
\usepackage{graphicx} % Images

\newcommand\blankpage{
    \null
    \thispagestyle{empty}
    \addtocounter{page}{-1}
    \newpage
    }
    
% quotes 
\newcommand{\quotes}[1]{``#1''}

\begin{document}
%Frontespizio
\begin{titlepage}
\begin{center}
    \large{UNIVERSITÀ DI PISA}
    \vspace{5mm}
\end{center}
%
%
\begin{figure}[!htb]
    \centering
    \includegraphics[width=5cm]{fig/logo.png}
\end{figure}
%
%
\begin{center}
    \vspace{2mm}
    \Large{DEPARTMENT OF COMPUTER SCIENCE}
    \\ \large{Master Programme in Data Science and Business Informatics}
    \vspace{3mm}
    \\ \LARGE{Master's Degree Thesis}
\end{center}
%
%
\vspace{5mm}
\begin{center}
    \Huge \bfseries
    {\LARGE{\bf Write your \\
           title here}}
\end{center}
%
\vspace{20mm}
%
\begin{minipage}[t]{0.47\textwidth}{
    \large{Supervisors}{
        \normalsize\vspace{3mm}\bf \\ 
        \large{
            Prof. Name Surname \vspace{2mm}\\
            Prof. Name Surname
        }
    }
}
\end{minipage}
\hfill
\begin{minipage}[t]{0.47\textwidth}\raggedleft
    {\large{Candidate}{\normalsize\vspace{2mm} \bf
    \\ \large{Name Surname\\ }}}
\end{minipage}


\vspace{25mm}
\centering{\large{Academic Year 2022/\the\year}}

\end{titlepage}
%Abstract
\begin{abstract}

\end{abstract}
%Indice
\tableofcontents

\chapter{Guidelines}

\begin{itemize}
    \item Rename the project as \quotes{Tesi\_Surname};
    \item Use a consistent formatting style throughout the document. This
          includes font size, line spacing, margins, and page layout. Use
          \textbf{only} full terms and avoid using abbreviations (e.g., cannot)
          in the text to ensure clarity and understanding for the reader. Use
          cross-referencing to link related sections, tables and figures;
    \item Use blank lines in order to separate paragraphs. \textbf{Avoid} using
          the \texttt{\textbackslash par} command or \textbackslash\textbackslash;
    \item After the third level in the Table of Contents, it is recommended to
          use the \texttt{\textbackslash subsection*\{\}} command instead of
          the numbered \texttt{\textbackslash subsection\{\}} command to avoid
          numbering the sub-sections and maintain consistency in the formatting
          of the document. For instance, you could have only Section 1, Section
          1.1 and Section 1.1.1;
    \item Use the following site for creating new tables:
          \url{https://www.tablesgenerator.com};
    \item Use \texttt{\$\dots\$}, e.g., $y = a^x$, to write in the
          \textit{inline} math mode. You can find other info here
          \url{https://it.overleaf.com/learn/latex/Mathematical_expressions}.
          An useful tool: \url{https://latex.codecogs.com/eqneditor/editor.php}.
    \item Use the following site for adding the references:
          \url{https://dblp.org}. You have to search for one of the author's
          names or the paper's title, click on the corresponding paper, and
          export the \textit{condensed} BibTeX. See screenshots in the fig
          folder in the project. Change the name of the reference using this
          format \texttt{\%s\%s\%s.format(FirstAuthorSurname, PublicationYear,
              KeywordTitle)} such as guidotti2021ensemble;
    \item It is recommended to use a tilde symbol near \texttt{\textbackslash cite\{\}} command to give more space between the text and the citation. This can improve the overall appearance and readability of the document.

\end{itemize}

\chapter{Introduction}
The introduction section is a crucial component in the overall structure and organization of the thesis.
Its primary purpose is to provide an overview of the research problem, its significance, and the research question(s) the study aims to answer. This section should be concise and to the point, captivating the reader's attention and generating interest in the research topic.

The introduction typically includes the following elements:
\begin{itemize}
    \item Background information on the research topic, providing context and setting the stage for the research to be presented;
    \item The statement of the research problem, highlighting the gap in knowledge that the research aims to fill;
    \item The significance of the research, explaining why the research is important and the impact it may have;
    \item A brief overview of the structure of the thesis, providing the reader with an outline of the main sections and chapters to be presented;
\end{itemize}



\chapter{Background}
A thesis's background section provides context and understanding of the presented research. It typically includes a review of relevant literature and research in the field, outlining the current state of knowledge on the topic and identifying gaps that the current research aims to fill. The background section also establishes the research problem's importance and relevance. It provides the reader with the necessary information to understand the context and significance of the research.


\chapter{State of the Art}
Citation: \cite{guidotti2018survey}.

\chapter{Methods}


\chapter{Experiments}


\chapter{Conclusions}
Here's an example structure for writing the conclusion of the thesis:

\begin{itemize}
    \item Restate the research question and summarize the main points of the thesis.

          Example: In this thesis, we have investigated how explainable artificial intelligence (XAI) can improve the transparency, accountability, and trustworthiness of AI systems. Through a combination of literature review and case studies, we have explored the benefits and challenges of XAI and identified several key factors that can impact its effectiveness;
    \item Summarize the findings and conclusions of the research.

          Example: Our research has shown that XAI can provide users with a better understanding of how AI systems work, increase their confidence in the system's decision-making process, and help identify and address potential biases and errors. However, implementing XAI can be challenging and requires careful consideration of factors such as the complexity of the AI model, the needs of the user, and the level of technical expertise required to interpret the explanations;

    \item Discuss the research implications and suggest areas for future study.

          Example: These findings have important implications for researchers, developers, and policymakers interested in promoting the responsible development and deployment of AI systems. Future research could focus on developing standardized frameworks and best practices for implementing XAI, as well as exploring the impact of XAI on different user groups and in different application domains;

    \item Conclude with a final statement or recommendation.

          Example: In conclusion, our research suggests that XAI has the potential to improve the transparency, accountability, and trustworthiness of AI systems, but its implementation requires careful consideration of a range of technical and human factors. By understanding the benefits and challenges of XAI, we can develop more effective strategies to promote the responsible development and deployment of AI systems that benefit society as a whole.
\end{itemize}

\bibliographystyle{plain}
\bibliography{biblio}
\end{document}

